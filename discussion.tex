\section{Discussion}
\label{sec:discussion} 
As is analyzed in Section~\ref{sec:experiment}, the \SA solver is basically a computation-intensive program while the requirement on cache and memory is not very high when text-based input file is adopted. From the statistics we could get the following insights:
\begin{compactitem}
\item Memory and branch/call instructions are both important contributors to low pipeline fetch efficiency. Besides more sophisticated branch prediction strategy should be further explored in future work.
\item Low ILP and MLP value implies that the \SA solver could behave well using in-order processors, which could save much expense on chip placement.
\item Floating point operations could not be substituted completely by fixed point ones.
\item \SA solver is basically computation-intensive rather than memory-intensive. Clock frequency and the performance of on-chip cache are still fundamental factors.
\end{compactitem}  