\section{Design}
\label{sec:design}
\subsection{Solver Overview}
The \SA polynomial solver is implemented in ANSI C language under GNU/Linux system with total 146 lines of code. The solver reads the input data from text files including the coefficients of the polynomial equations and calculate the optimal solution (extreme point) of each equation using \SA with given initial and final temperature. The result of each equation would be recorded in the output files.
\subsection{Input Data}
A typical record in the input data file is like this:
$$[-9397.819, 602.181] -799.515 -687.799 -411.936$$
The two numbers in the brackets define the domain of the extreme point, i.e. the solver would only search the optimal solution within this domain. The following several numbers define the coefficients in descending order. For instance, the record above indicates an equation of:
$$f(x) = -799.515x^2-687.799x-411.936$$
where $x\in[-9397.819, 602.181]$. Similarly, the format of the output is like this:
$$x = -0.986, cost(x) = -2.928e+03$$
This means the optimal solution and the corresponding cost of the function.
\subsection{Parameters}
The usage of the main program is as follows:
\begin{verbatim}
./anneal <input_file> <output_file> 
  [--options]
\end{verbatim}
The meaning of each domain is explained as follows:
\begin{compactitem}
\item input\_file: Path of the input data file
\item output\_file: Path of the output data file
\item options: Optional parameters of the program including
    \begin{compactitem}
    \item ini\_temp: The initial temperature; default value is 1000.
    \item fin\_temp: The final temperature; default value is 5e-5.
    \item lambda: The coefficient of temperature decrease; default value is 0.7.
    \end{compactitem}
\end{compactitem}
\subsection{Comparing Version}
As a fixed-point version for comparison, I substitute the floating point operations in the \SA solver by fixed-point ones and adjust the bits of decimal part. The input data is the same with floating point ones while the output result would be evaluated according to the criteria that would be introduced in Section~\ref{sec:comparison}. 